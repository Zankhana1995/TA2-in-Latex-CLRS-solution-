\documentclass[a4paper, 11pt]{article}
\usepackage{comment} % enables the use of multi-line comments (\ifx \fi) 
\usepackage{fullpage} % changes the margin
\usepackage{hyperref}
\usepackage{amsmath}
\usepackage{amssymb}
\usepackage{booktabs} % For formal tables

\usepackage[ruled]{algorithm2e} % For algorithms
\renewcommand{\algorithmcfname}{ALGORITHM}

\begin{document}
%Header-Make sure you update this information!!!!
\noindent
\large\textbf{Homework TA2} \hfill \textbf{Zankhanaben Ashish Patel} \\
\normalsize COMP6651 \hfill \textbf{40067635} \\
Prof. Tiberiu Popa \hfill Due Date: 12/04/19 


\section{22.2 - 8}
Let us Assume that $x$ and $y$ are 2 endpoints of the path in tree t who has the diameter. Let us assume random vertex in tree T called $s$ such that distance from s to $x$ and $y$ are greatest. So that we can generate diameter from vertices $x$, $y$ and $s$. \\
We will prove above statement with proof of contradiction. Let's assume one vertex $p$ , which is not on the path of $x$ and $y$. The distance from $s$ to $p$ is  farthest.
\begin{equation}
\label{eq1}
\delta(s,x) \le \delta(s,p)
\end{equation}
\begin{equation}
\label{eq2}
\delta(s,y) \le \delta(s,p)
\end{equation}
Let's assume there is another vertex $c$ on the path of $x$ and $y$ and $c$ is on the tree, which minimizes $\delta(s,c)$.\\
We will have  $\delta(s,x) = \delta(s,c) + \delta(c,x).$\\
Same way $ \delta(s,y) = \delta(s,c) + \delta(c,y).$\\
As vertex $c$ is on path $x$ and $y$ we can say that,
\begin{equation}
\label{eq3}
\delta(c,x) + \delta(c,y) = \delta(x,y)
\end{equation}
$$\delta(x,y) = \delta(s,x) + \delta(s,y)$$
\begin{equation}
\label{eq4}
\delta(x,y) = \delta(s,c) + \delta(c,x) + \delta(s,c) + \delta(c,y)
\end{equation}
from (\ref{eq3}) and (\ref{eq4}), we can say that,
\begin{equation}
\delta(s,x) + \delta(s,y) = 2 \delta(s,c) + \delta(x,y) 
                          = \delta(s,c) + \delta(c,x) + \delta(s,c) + \delta(c,y)
\end{equation}
$$\therefore \delta(s,x) + \delta(s,y) \leq \delta(s,c) + \delta(c,x) + \delta(s,c) + \delta(c,y)$$
$$\therefore \delta(x,y) \leq \delta(x,y) + 2 \delta(s,c)$$
From (\ref{eq4}) we can write that,
$$\delta(x,y) = \delta(s,x) + \delta(s,c) + \delta(c,y)$$
From (\ref{eq1}) we know that $\delta(s,x) \le \delta(s,p)$
$$\delta(x,y) \leq \delta(s,p) + \delta(s,c) + \delta(c,y)$$
Now let's take $\delta(p,y) = \delta(s,c) + \delta(s,y).$ to prove that point $p$ is not on the path of diameter, though it's longest. If this equation is not true means there is no path from $p$ to $y$ which is going from $c$, and this line give us a contradiction as we have already assumed that $c$ is there in the tree.\\
$\therefore$ The inequality of
$$\delta(x,y) \le \delta(x,y) + 2 \delta(s,c) \le \delta(p,y)$$
will be false.\\
So our assumption of either $a$ or $b$ or both as the whose distance from $s$ is greatest. So $\delta(x,y)$ is maximal among all pairs.\\
We can solve this problem by BFS. Running BFS again and again will give us the path from $x$ to $y$ and we can derive that which one greater in $O(V)$ time. Hence, we can find diameter.

\section{22.3 - 13}
Using DFS we can solve this problem, we have to make one small change in original DFS algo inside for loop of line-4 of $DFS-VISIT(G,u)$ procedure.\\
Idea : When we are exploring adjacent edges of any vertex, if it is visited already by previous vertex means $v.color != white $ (it's grey or black) than, we can assume that another path is also exist.We can say that $G(V,E)$ is not singly connected.

\vspace{5mm}
\noindent
$DFS(G)$\\
1. for each vertex $ u  \in  G.V $\\
2. \hspace{0.7cm} $u.color = white$\\
3. \hspace{0.7cm} $u.\pi = NIL$\\
4. time $ = 0$\\
5. for each vertex $ u  \in  G.V $\\
6. \hspace{0.7cm} if $u.color = white$\\
7. \hspace{1.4 cm} $DFS-VISIT ( G,u )$

\vspace{5mm}
\noindent
$DFS-VISIT(G,u)$\\
1. time = time + 1\\
2. $u.d = $ time\\
3. $u.color = grey$\\
4. for each $v \in G.adj[u]$\\
5.\hspace{0.7cm}$v.color == white$\\
6.\hspace{1.4cm}$v.\pi = u$\\
7.\hspace{1.4cm}$DFS-VISIT(G,v)$\\
8.\hspace{0.7cm}else\\
9.\hspace{1.4cm} print "Not singly connected"\\
10.\hspace{1.4cm}break\\
11. $u.color = black$\\
12. time = time + 1\\
13. $u.f = $ time\\
(Reference : CLRS textbook)

\vspace{5mm}
\noindent
The slight change I have made at line 8 to 10 (the else part). This part will be executed when our graph will not singly connected. Otherwise we'll get output of our traditional DFS.

\section{23.1 - 11}
We can this problem by kruskal or Prim's algorithm by making some change in it. If we decrease the weight of one edge in the graph and if we will add that edge in the graph than it'll create cycle we will break the rule of existing algorithm. To overcome from this issue, consider given algorithm after adding new edge into tree.

\vspace{5mm}
\noindent
1. Create set A from vertices of tree.\\
2. Sort the edges of $G.E$ in decreasing order by weight w.\\
3. Remove edge $(u,v)$ with max weight w.\\
4. Return A.

\vspace{5mm}
\noindent
This algorithm will take same time as kruskal 's algorithm which is $O(E\lg V) $ 
time.

\section{26.3 - 4}
Neighbourhood of $X$ is defined as,
$N(X) = \{ y \in V : (x,y) \in E ,x \in X\}$. and we have to find that there exist a perfect matching in G if and only if $|A| \leq |N(A)|$ for every subset $A \subseteq L$.\\
We will start with proving $|A| \leq |N(A)|$ for every subset $A \subseteq L$.\\
Since it's matching between L and R, no two vertices have matching between themselves in R, therefore, at least $|A|$ vertices in neighborhood of A which are matched with R.\\
The number of augmented path you will find the flow will increase each time (Corollary 26.3 of CLRS textbook). \\
For the sake of this problem if we will assume the capacity of each edge is one than flow will increase by 1 at the end of every run of ford-fulkrson (FF) algorithm. The flow should be strictly integer (Theorem 26.10 CLRS textbook).\\
Consider given step as the proof of Hall's theorem, ( Note- FF : ford-fulkrson algorithm)

\vspace{5mm}
\noindent
1. Run FF on $G(V,E)$ and take residual graph $G'(V',E')$ where,\\
$V' = V \cup \{s\} \cup \{ t\} $\\
$E' = \{ (s,u) : u \in L \} \cup \{(u,v) : (u,v) \in E \} \cup \{ (v,t) : v \in R \}.$ (Reference CLRS textbook)\\
and $L' = L \cup \{ s\} $, $R' = R \cup \{ t\}$

\vspace{5mm}
\noindent
2. Every time when you run FF the flow $f$ will be increased by 1 and we'll get match of one vertex $u \in L$ to corresponding $v \in R$, So no other vertex $u' \in L$ can connect to v which gives us perfect matching.

\vspace{5mm}
\noindent
3. On the next run, if we want to find match of $u' \in L$, we have to choose augmented path which does not contain v means $v' \in R - \{ v \}$ where v' will be perfect matching for u'.

\vspace{5mm}
\noindent
4. If the flow value $|f|$ at the end is equal to $|L|$ than we can say $\forall  u \in L $ has perfect matching.

\vspace{5mm}
\noindent
5. Therefore, $A \subseteq L$ has at least $|A| $ neighbors and $|A| \leq |N(A)|$ so that there exist a perfect match in $G$.
\end{document}